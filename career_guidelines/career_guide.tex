\documentclass[11pt]{article}

\usepackage{fullpage}
\usepackage{listings}
\usepackage{multicol}

\begin{document}
\pagestyle{empty}

\begin{center}
University of Notre Dame\\
Department of Computer Science and Engineering\\
\vspace{.5in}
{\bf Project 2}\hspace{\fill}\parbox[b]{3in}{\centering{\bf
    Ethics}}\hspace{\fill}{\bf Shane Ryan}
{\centering\makebox[\textwidth]{\hrulefill}\hfill}

\vspace{5pt}\centering{\bf Career and Internship Guidelines\\
How to Stay on Top of Career Development}

{\centering\makebox[\textwidth]{\hrulefill}\hfill}

\end{center}


\begin{enumerate}


\item \textbf{When should students start preparing or planning for internship
    or job interviews?}

It's never too early to start preparing for interviews!  But don't be scared -
you may already be preparing right now without knowing.  Many interviews will
        test technical knowledge and comprehension which are furthered by
        staying attentive in class and sharpening your skills outside of the
        classroom.

The best way to avoid stressful or rushed preparation for an interview is by
        working at it in small steps well before landing such an interview.
        First-Year students shouldn't worry until they have contact with a
        company that has a possible opportunity.  As a Sophomore one should
        plan further ahead because of the increased workload faced throughout
        the semester, and by Junior year you should have had a few interviews
        or at least a few practice ones with the Career Center.

\item \textbf{How should students prepare or plan for these interviews?}

Working on projects that contain work within the scope of your field is one of
the best ways to prepare for these interviews.  You will be able to point to
        something tangible outside of the classroom that demonstrates your
        passion for and thorough understanding of material.  Going beyond
        studying material for an exam and actually putting it to practical use
        will help you cover the small details that may be harped upon during an
        interview as well.

Reading sections of interview books that cover technical details as you learn
        the same technical details in class will help solidify topics.  For
        example, building a linked-list for an assignment in C++ may be of no
        use to you in an interview if they ask to work in a different language
        and you don't understand the underlying motives of creating a
        linked-list.  Understanding syntax is important, but understanding
        concepts is imperative - some companies may even give you a command set
        for a language they created in order to test your ability to adapt and
        learn.

\item \textbf{What resources should students consider? Books? Career Services?
    Student groups?}

This will vary for each situation:

Student groups are the best long-term solution to being prepared for a
        technical interview.  The knowledge you gain through applied
        experiences will outweigh cramming an interview book in the long run.

If you are not familiar with the setup of an interview a mock interview at the
Career Center will be very helpful.  Meeting with someone who can give concrete
feedback on your performance during the interview can help give you direction
        on further preparation.  After doing well in this setting, students are
        more likely to improve by honing in on particular topics using
        interview books.

These books are most helpful for improving upon specific topics and reviewing
        in a general sense in the weeks and days immediately before an
        interview.  You should not be seeing the information in these books for
        the first time - a solid base in the classroom and experience through
        groups or clubs will give you first exposure in a much more meaningful
        way.

\item \textbf{What extracurricular activities should students consider?}

Anything related to a specific type of job you would like to be a part of in
the future should be your highest priority - for example, a game development
club if you are seeking employment in the entertainment or gaming sector.  If
no such relevant club exists for your interests, consider starting one yourself
with a few fellow students, or take on a role in another club where you can
best apply those relevant skills.

Be careful to not let the time committment of a club or activity to dominate
your school experience though - learning in the classroom should be the
priority.

Interdisciplinary clubs that draw from multiple majors will be more beneficial
for networking with other people!  Sometimes getting your foot in the door for
an interview with a company can stem from something as simple as knowing
another club member who has a connection, or can speak to your abilities.

\item \textbf{How can students take advantage of networking and alumni
    relationships?}

Connect with upperclassman in your dorm and in your major in order to start
building your network.  These students have more experience and have likely
built up connections which they can share with you.

The career center has an alumni networking tool that acts as a searchable
database to list contact info of graduates.  Filtering by companies of
interest, field of study, even graduating class or dorm is a powerful way to
find receptive alumni with which you have things in common.  

If a few alumni do not respond, don't hesitate to reach out to more or send a
friendly reminder at a later date.

Attending on campus recruiting opportunities for companies will help you get in
touch with graduates who represent their employer and can help you bridge the
gap between school and full-time jobs.

\item \textbf{How should students approach negotiations or contracts? Are there
    any pitfalls they should look out for?}

Negotiating a contract may seem like a daunting task, but when approached
conservatively is a very useful method for making sure you receive the best
possible offer.  It is always a good idea to politely show employers a
competitors offer if you are interested in having it matched.  Only do such a
thing once a full-time offer has been extended, and do so graciously and
respectfully.  Many employers will happily match another offer in order to
retain you or at least further incentivize you to join their company.

\item \textbf{Anything else you wish you knew before you went through the whole
    process!}

Don't be afraid to reach out to professors!

Some of the most fruitful internship advice and connections I have had to date
came from having healthy relationships with professors.  If a topic or class
interests you, let the professor know, and see if they have any points of
contact in industry or in research areas to which you can contribute.

Also, you will hear many 'No's, so don't be discouraged and keep on applying!

\end{enumerate}


\end{document}


